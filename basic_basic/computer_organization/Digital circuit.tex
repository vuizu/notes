\documentclass{template}

\begin{document}
\ttfamily
\section{数字电路基础}
    计算机硬件发展史:电子管 -> \key{晶体管}{(transistor)} -> 中小规模集成电路 -> 超大规模集成电路

    晶体管分为两类:单极性晶体管和双极性晶体管,其中单极性晶体管又分为两类:绝缘栅型场效应管(IGFET),也就是MOS管(Metal-Oxide-Semiconductor,金属氧化物半导体),
    和结型场效应管(JFET),


    场效应管(Field-Effect Transistor, FET):基于半导体材料的电子器件,利用外加电场来控制电荷载流子在半导体中的运动,从而实现电流的控制。

    MOS管又可以分为PMOS、NMOS和CMOS,他们是门电路的基本组成单元。

    \subsection{半导体}
        金属导体中,金属原子容易失去最外层电子,从而形成阳离子,这些阳离子在金属中形成了一个紧密排列的晶体结构,离开原子的价电子并不专属于某一特定的阳离子,
        而是在电子云中自由移动,成为与若干阳离子相吸引的电子,金属阳离子和自由电子相互吸引而将所有阳离子结合在一起的方式就是金属键。
            
        无论是导体还是半导体,其中都存在着大量可自由移动的带电粒子,在外加电场的作用下,载流子做定向运动,从而形成电流。
        导体中的载流子为\key{自由电子}{};半导体中的载流子为\key{自由电子}{}和\key{空穴}{};溶液中的载流子为\key{阴离子}{}和\key{阳离子}{}。

        \subsubsection{电源}
            我们以铜锌干电池为例,Zn做负极,稀$H_2SO_4$作为电解质溶液,将正负极放入溶液中后,
            由于自由电子是无法在溶液中移动的,所以导致负极电势变低,从而与正极形成电势差,


            接通电路后,导线中的电子定向移动形成电流,而负极产生的电子也随着定向移动,补充导线中消耗的电子,同时溶液的阴离子往负极移动,和负极的阳离子生成盐,
            溶液的阳离子往正极移动,和运输到正极的电子结合。



        \subsubsection{什么是空穴?}
            空穴可以被看作是一个带有正电荷的粒子。

            为什么使用半导体而不是用导体的原因。
            电子和空穴的数量可以通过控制掺杂杂质的类型和浓度来调节,从而调整半导体的电导率。而导体材料的电导率很高,不能通过控制来实现特定的电性能和功能。

            半导体器件中的电子和空穴的移动速度比导体中的电子流速度要快得多,这使得半导体器件可以实现更快的开关速度和更低的功耗。


            往半导体中添加5价磷,就变成N(Negative)型半导体 -- 多了电子;添加3价硼,就变成了P(Po\\sitive)型半导体 -- 多了空穴;


    \subsection{晶体管}
        这是最基础也是最核心的电子元件,通常是由\key{半导体}{(Semiconductor)}材料制作而成,利用了半导体材料中的电子和空穴的特性
        \footnote{特性:我}来控制电流的流动,
        主要分为两大类:\key{双极性结型晶体管}{(Bipolar Junctio\\n Transistor, BJT)}
        和\key{场效应晶体管}{(Field-Effect Transistor, FET)}。

        \textit{BJT}通常指的是三极管,它的三个极为\key{发射极}{(Emitter)}、\key{基极}{(Base)}和\key{集电极}{(Collector)},
        是由2个PN结组成,
        主要有PNP型和NPN型; 

        后者通常是指MOS场效应管,有PMOS和NMOS两种类型,可以由两者组合成一种电路设计技术CMOS,CMOS并不是一个具体的电子器件。

        半导体中的载流子有两种,分别是带负电的自由电子和带正电的自由空穴。

\end{document}