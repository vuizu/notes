\documentclass{template}

\begin{document}
\ttfamily
存储器系统是一个具有不同容量、成本和访问时间的存储设备的层次结构。CPU寄存器保存着最常用的数据。
靠近CPU的小的、快速的\key{高速缓存存储器}{(Cache Memory)}作为一部分存储在相对慢速的
\key{主存储器}{(Main Memory)}中数据和指令的缓冲区域。主存缓存存储在容量较大的、慢速磁盘上的数据,
而这些磁盘常常又作为存储在通过网络连接的其他机器的磁盘或磁带上的数据的缓冲区域。

存储器按照存取方式可以分为三部分:\key{随机存储器}{RAM}、\key{只读存储器}{ROM}和\key{串行访问存储器}{}。

\section{寄存器}
    这里的寄存器不是指单一的寄存器,在CPU内部分布着多个寄存器。


\section{随机访问存储器}
    \key{随机访问存储器}{(Random-Access Memory, RAM)}分为两类:静态的和动态的,静态比动态更快,
    但容量较小,也贵得多。SRAM用来作为高速缓存存储器,既可以集成在CPU芯片上,也可以在片下。
    DRAM则用来作为主存及图形系统的帧缓冲区。

    只要有供电,SRAM就会保持不变,与DRAM不同,它不需要刷新。SRAM的存取比DRAM快。SRAM对诸如光和电噪声这样的
    干扰不敏感,代价是SRAM单元比DRAM单元使用更多的晶体管,因而集成度低,而且更贵,功耗更大。

    \begin{table}[!h]
        \centering
        \begin{tabular}{c|c|c|c|c|c|c}
            \hline
                & 每位晶体管数 & 相对访问时间 & 是否持续 & 是否敏感 & 相对花费 & 应用 \\
            \hline
                \textit{SRAM} & 6 & 1X & 是 & 否 & 1000X & 高速缓存存储器 \\
            \hline
                \textit{DRAM} & 1 & 10X & 否 & 是 & 1X & 主存,帧缓冲区 \\
            \hline
        \end{tabular}
        \caption{DRAM和SRAM存储器的特性}
    \end{table}


    \subsection{静态RAM}
        SRAM将每个位存储在一个\key{双稳态的}{(bistable)}存储器单元里。每个单元是用一个六晶体管
        \footnote{\href{https://blog.csdn.net/SUPREME_SYZ/article/details/124642319}
        {晶体管:泛指一切基于半导体材料制作的单一元件}}电路来实现的。
        这样的电路可以无限期地保持在两个不同的电压配置或状态之一。其他任何状态都是不稳定的--从不稳定状态开始,
        电路会迅速地转移到两个稳定状态中的一个。采用了触发器原理来实现存储,每个存储单元包含两个互补的触发器。

        由于SRAM存储器单元的双稳态特性,只要有电,他就会永远地保持它的值,即使有干扰(例如电子噪音)来扰乱电压,
        当干扰消除时,电路就会恢复到稳定值。

        缓存全称高速缓冲存储器,位于主存和CPU之间,用来存放正在执行的程序段和数据,以便CPU能高速地使用它们。
        全部集成在现代CPU内部。

    \subsubsection{局部性原理}
        局部性通常有两种不同的形式:\key{时间局部性}{(temporal locality)}和\key{空间局部性}{(spatial locali
        \\ty)}。在一个具有良好时间局部性的程序中,被引用过一次的内存位置很可能在不远的将来再被多次引用。在一个具有良好
        空间局部性的程序中,如果一个内存位置被引用了一次,那么程序很可能在不远的将来引用附近的一个内存位置。

        在硬件层,局部性原理允许引入称为\key{}{Cache}的存储器来保存最近被引用的指令和数据项,从而提高对主存的访问速度。
        在操作系统层,局部性原理允许系统使用主存作为虚拟地址空间最近被引用块的高速缓存,用主存来缓存磁盘文件系统中
        最近被使用的磁盘块。




    \subsubsection{高速缓存 Cache}
        CPU从内存中读取数据到Cache的过程中,是一小块一小块来读取数据的,而不是按照单个元素来读取数据的。这样一小块
        一小块的数据,在\key{}{Cache}里面,叫做\key{缓存行}{(Cache Line)}。\key{}{Cache Line}的大小通常是
        64字节。






    \subsection{动态RAM}
        DRAM存储器中每个单元由一个电容和一个访问晶体管组成。

        DRAM将每个位存储为对一个电容(非常小)的充电。动态RAM(以电容充电原理寄存信息).

        用来存放计算机运行期间所需的大量程序和数据,CPU可以直接随机地其进行访问,也可以和Cache及辅存交换数据。




\section{只读存储器}
    ROM和RAM的存取方式均为随机存取。信息一旦写入存储器就固定不变,即使断电,内容也不会丢失。
    它与RAM可共同作为主存的一部分,统一构成主存的地址域。




\section{串行访问存储器}
    \subsection{磁盘}
        磁盘由\key{盘片}{(platter)}构成,每个盘片有两面,每面覆盖着磁性记录材料,盘片中央有一个可以旋转的
        \key{主轴}{(spindle)},它使得盘片以固定的\key{旋转速率}{(rotational rate)}旋转。磁盘通常包含
        一个或多个这样的盘片,并封装在一个密封的容器内。

        每个表面是由一组称为\key{磁道}{(track)}的同心圆组成。每个磁道被划分成一组\key{扇区}{(sector)},
        每个扇区包含相等数量的数据位(通常是512字节),这些数据编码在扇区上的磁性材料中。扇区之间由一些\key{间隙}
        {(gap)}分隔开,这些间隙中不存储数据位。间隙存储用来标识扇区的格式化位。

        磁盘是由一个或多个叠放在一起的盘片组成,它们被封装在一个密封的包装里,整个装置通常被称为\key{磁盘驱动器}
        {(disk drive)},简称磁盘。有时我们会称磁盘为旋转磁盘,以使之区别于基于闪存的固态硬盘(SSD),SSD是没有移动部分的。

        柱面是所有盘片表面上到主轴中心的距离相等的磁道的集合

    \subsection{固态硬盘SSD}
        SSD和HDD




\section{虚拟存储器}
    它允许操作系统将正在执行的进程所需的部分数据存储在主存储器(RAM)中,而将其他部分数据存储在磁盘等其他存储介质中。
    这样可以让操作系统看起来像拥有更多的内存,从而提高系统的性能和可靠性。


\end{document}
