\documentclass{template}


\begin{document}
\ttfamily

Java作为一门高级程序语言,它的语法非常复杂,抽象程度也很高。因此直接在硬件上运行这种复杂的程序并不现实。

\section*{Java虚拟机具体是怎样运行Java字节码的?}
    执行Java代码首先需要将它编译而成的class字节码文件加载到Java虚拟机中。加载后的Java类会被存放于方法区。

    Java虚拟机同样也在内存中划分出堆和栈来存储运行时数据。Java虚拟机会将栈细分为面向Java方法的Java方法栈,面向本地方法(用C++写的native方法)
    的本地方法栈,以及存放各个线程执行位置的PC寄存器。

    在运行过程中,每当调用进入一个Java方法,Java虚拟机会在当前线程的Java方法栈中生成一个栈帧,用以存放局部变量以及字节码的操作数。这个栈帧
    的大小是提前计算好的,而且Java虚拟机不要求栈帧在内存空间里连续分布。

    当退出当前执行的方法时,不管是正常返回还是异常返回,Java虚拟机均会弹出当前线程的当前栈帧,并将之舍弃。

    Java字节码无法直接执行,因此JVM需要将字节码翻译成机器码。在hotspot里,上述翻译过程有两种形式:
    \begin{itemize}
        \item 第一种是解释执行,即逐条将字节码翻译成机器码并执行;
        \item 第二种是\key{即时编译}{(Just-In-Time compilation, JIT)},即将一个方法中包含的所有字节码编译成机器码后再执行。
    \end{itemize}
    JVM会先解释执行成字节码,而后将其中反复执行的热点代码,以方法为单位进行即时编译。

    hotspot内置了多个即时编译器:C1、C2和Graal。引入多个即时编译器的原因是为了在编译时间和生成代码的执行效率之间进行取舍。C1是Client编译器,
    采用的优化手段相对简单,因此编译时间较短;C2是Server编译器,采用的优化手段相对复杂,因此编译时间较长。

    从JDK7开始,hotspot默认采用分层编译的方式:热点方法首先会被C1编译,而后热点方法中的热点会进一步被C2编译。为了不干扰应用的正常运行,hotspot
    的即时编译是放在额外的编译线程中进行的,hotspot会根据CPU的数量设置编译线程的数目,并且按1:2的比例配置位C1及C2编译器。

\section{}

\end{document}